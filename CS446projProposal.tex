\documentclass[11pt,letterpaper]{article}
\oddsidemargin 0in
\evensidemargin 0in
\textwidth 6.5in
\topmargin -0.5in
\textheight 9.0in
\usepackage{hyperref}
\usepackage{mathptmx}
\usepackage{graphicx}
\usepackage[usenames,dvipsnames]{xcolor}
\newcommand{\blue}[1]{\textcolor{RoyalBlue}{#1}}
\newcommand{\fillme}[1]{\blue{\texttt{[Insert #1]}}}
\newcommand{\instructions}[1]{\blue{\textit{#1}}}
% uncomment the next two lines if you want the instructions to disappear.
%\renewcommand{\instructions}[1]{}
%\renewcommand{\fillme}[1]{}

\begin{document}

\title{CS446 Class Project Proposal \fillme{Your Project}}
\author{\fillme{Your Names (Your NetIDs)}}
\maketitle



\instructions{If you are taking CS446 for 4 hours credit, you need to
  do a research project.\\
This is a \LaTeX template for the initial proposal,  but should also give you a start on the final report.\\
The blue pieces of text  in this template are either instructions ({\tt$\backslash$instructions\{...\}}) or indicate where you need to fill in something ({\tt$\backslash$fillme\{...\}}).  
You should replace all the {\tt$\backslash$fillme\{...\}} commands with your own text.
To make the instructions disappear, please uncomment the 
\begin{center}
{\tt$\backslash$renewcommand\{$\backslash$instructions\}[1]\{\}}\\
%{\tt$\backslash$renewcommand\{$\backslash$fillme\}[1]\{\}}\\
\end{center}
lines in the preamble (just above  {\tt $\backslash$begin\{document\}} of this .tex file) by removing the leading \% marks, 
recompile (run \LaTeX again) and submit the PDF on Compass.\\
The template for the final report is at
\url{http://courses.engr.illinois.edu/cs446/Projects/CS446proj.tex}
(or
\url{http://courses.engr.illinois.edu/cs446/Projects/CS446proj.pdf}
for the pdf)
}
\section*{Task description}
\instructions{Describe the task you want to tackle in your project.}

\section*{Background}
\instructions{Has there been any prior work on this task? If so,
  provide references where available}

\section*{Data and evaluation}
\instructions{Do you have data to train and test your system on? How
  will you evaluate your system?}

\section*{Your approach} 
\instructions{Describe how you want to tackle this task}

\section*{Your to-do list}
\instructions{Get started by making a to-do list. If you have a group
  project: who will do what? Set yourself deadlines. Here are a few
  items that might appear on your to-do list}
\begin{enumerate}
\item \fillme{...do you have data?}
\item \fillme{...do you know related work? (have you got the
    references  for your .bib file?)}
\item \fillme{...what algorithm will you use? do you need to implement
    this yourself, or will you use an off-the-shelf package?} 
\item \fillme{...what experiments do you plan to run?}
\item \fillme{...and don't forget to allocate time for the writeup!} 
\end{enumerate}

\section*{Bibliography}
\instructions{If you need references for the background section, don't forget to create your own .bib file, and run {\tt bibtex}. If you call your bibliography {\tt mybib.bib} and put it in the same directory as this {\tt .tex} file, add {\tt$\backslash$bibliography\{mybib\}} before {\tt$\backslash$end\{document\}}
}
\bibliographystyle{plain}  
\end{document}

%%% Local Variables:
%%% mode: latex
%%% TeX-master: t
%%% End:

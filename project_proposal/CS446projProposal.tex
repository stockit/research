\documentclass[11pt,letterpaper]{article}
\oddsidemargin 0in
\evensidemargin 0in
\textwidth 6.5in
\topmargin -0.5in
\textheight 9.0in
\usepackage{hyperref}
\usepackage{mathptmx}
\usepackage{graphicx}
\usepackage[usenames,dvipsnames]{xcolor}
\usepackage{mathtools}
\newcommand{\blue}[1]{\textcolor{RoyalBlue}{#1}}
\newcommand{\fillme}[1]{\blue{\texttt{[Insert #1]}}}
\newcommand{\instructions}[1]{\blue{\textit{#1}}}
% uncomment the next two lines if you want the instructions to disappear.
\renewcommand{\instructions}[1]{}
\renewcommand{\fillme}[1]{}

\begin{document}

\title{CS446 Class Project Proposal - Stockit}
\author{dmcquil2@illinois.edu, mcconne7@illinois.edu}
\maketitle



\instructions{If you are taking CS446 for 4 hours credit, you need to
  do a research project.\\
This is a \LaTeX template for the initial proposal,  but should also give you a start on the final report.\\
The blue pieces of text  in this template are either instructions ({\tt$\backslash$instructions\{...\}}) or indicate where you need to fill in something ({\tt$\backslash$fillme\{...\}}).  
You should replace all the {\tt$\backslash$fillme\{...\}} commands with your own text.
To make the instructions disappear, please uncomment the 
\begin{center}
{\tt$\backslash$renewcommand\{$\backslash$instructions\}[1]\{\}}\\
%{\tt$\backslash$renewcommand\{$\backslash$fillme\}[1]\{\}}\\
\end{center}
lines in the preamble (just above  {\tt $\backslash$begin\{document\}} of this .tex file) by removing the leading \% marks, 
recompile (run \LaTeX again) and submit the PDF on Compass.\\
The template for the final report is at
\url{http://courses.engr.illinois.edu/cs446/Projects/CS446proj.tex}
(or
\url{http://courses.engr.illinois.edu/cs446/Projects/CS446proj.pdf}
for the pdf)
}
\section*{Task description}
  Stockit is an algorithm developed to forecast stock price changes from
  a given news article body. A successful algorithm will be one that provides
  a statistically significant advantage to predicting the percentage change
  in a stock price.
\instructions{Describe the task you want to tackle in your project.}

\section*{Background}
\instructions{Has there been any prior work on this task? If so,
  provide references where available}

\section*{Data}
  Our data will consist of a collection of news articles and historic
  stock price information. News articles were gathered using a web crawler.
  Historic stock prices were gathered using the
  \href{https://developer.yahoo.com/yql}{yql api}.
  
\section*{Assumptions}
  \begin{itemize}
  \item News articles lead stock price changes.
  \item A given news article relates to specifically $n$ publically traded companies.
  \end{itemize}
  Note: These assumptions are not necessarily accurate.
  
\instructions{Do you have data to train and test your system on? How
  will you evaluate your system?}
  
\section*{Evalutation}


\section*{Our approach}
  We will separate our news articles into test and training sets. For the training news
  articles, we will find the $n$ companies referred to in the article by taking a given
  list of companies and checking it is "matched" in the article. For example, we will
  take company names and their aliases and check for text similarity score for a given
  document. Once an article is mapped to a document, we will look up the $\%\delta$ in
  the stock price on the day the article was published. Accordingly, we will remove references
  to the company from the body of the document. For given test document the $\%\delta$
  will be calculated by using the K nearest documents using text similarity
  and taking the median $\%\delta$ of those documents.
\instructions{Describe how you want to tackle this task}

\section*{Your to-do list}
\instructions{Get started by making a to-do list. If you have a group
  project: who will do what? Set yourself deadlines. Here are a few
  items that might appear on your to-do list}
\begin{enumerate}
\item Retrieve news articles and historic stock price data.
\item We have not currently looked up any related work.
\item We will use am off the shelf KNN implementation.
\item We plan to check the validity of the algorithm and also explore some possible
  extensions. What if document map to an industry instead and stock are also related
  to industries?
\item Finish write up.
\end{enumerate}

\section*{Bibliography}
\instructions{If you need references for the background section, don't forget to create your own .bib file, and run {\tt bibtex}. If you call your bibliography {\tt mybib.bib} and put it in the same directory as this {\tt .tex} file, add {\tt$\backslash$bibliography\{mybib\}} before {\tt$\backslash$end\{document\}}
}
\bibliographystyle{plain}  
\end{document}

%%% Local Variables:
%%% mode: latex
%%% TeX-master: t
%%% End:
